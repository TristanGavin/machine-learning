\documentclass{article}

% Formatting
\usepackage[utf8]{inputenc}
\usepackage[margin=1in]{geometry}
\usepackage[titletoc,title]{appendix}
\usepackage{listings}


% Math
\usepackage{amsmath,amsfonts,amssymb,mathtools}
\usepackage[ruled,vlined]{algorithm2e}
\usepackage{algorithmic}

% Images
\usepackage{graphicx}

% Tables
\usepackage{booktabs}

% References
\usepackage{biblatex}
\addbibresource{references.bib}

% Title content
\title{CS 434 HW0}
\author{Tristan Gavin}
\date{March 31, 2021}

\begin{document}

\maketitle

\section*{Probability}
\subsection*{1. Bayes Theorem and Marginalization}
Information we are given:
\begin{itemize}
    \item rains 73 days each year (365 days)
    \item Weatherperson is 70\% accurate forcasting rain when it rains
    \item Weatherperson innacurately predicts rain 30\% of the time when it does not rain
  \end{itemize}
Lets create the set $R \in \{0,1\}$ for when it doesn't rain and when it does rain. The set $F \in \{0,1\}$ for whether the weatherperson doesn't predict rain or does predict rain respectively. We want to find the probability that it will rain given that the weatherperson predicted rain (denoted $P(R=1 | F=1)$). To solve this we will implement Bayes Theorem \[ P(R=1|F=1) = \frac{P(F=1|R=1)P(R=1)}{P(F=1|R=1)P(R=1) + P(F=1|R=0)P(R=0)} \]

\noindent Fortunately we are given enough information to fill out this equation. 
\begin{itemize}
    \item[] $P(F=1|R=1) = 0.7$ (probability that the forcaster predicted rain given that it rains)
    \item[] $P(R=1) = 73 \div 365 = 0.2$ (probability of rain on any given day)
    \item[] $P(F=1|R=0) = 0.3$ (Probability that weatherperson innacurately predicts rain given it does not rain) 
    \item[] $P(R=0) = 1 - (73 \div 365) = 0.8$ (probability that it doesn't rain on any given day)
  \end{itemize}

\[P(R=1 | F=1) = \frac{(0.7)(0.2)}{(0.7)(0.2)+(0.3)(0.8)} = \frac{0.14}{0.38} \approx 0.3684\]

\noindent There is a 36.84\% chance that it will rain tomorrow

\subsection*{2. Computing Expected Values from Discrete Ditributions}
There is a $P(x=1) = \frac{1}{6}$ chance of rolling a one and $p(x\neq1)=\frac{5}{6}$ chance of not rolling a 1. Taking the expected values of all sample events and multiplying them by the amount of money we will make or lose if the event is to happen we have $P(x=1)(1)+P(x=2)(-.25)+P(x=3)(-.25)+P(x=4)(-.25)+P(x=5)(-.25)+P(x=6)(-.25)= 1/6(1.25)+1/6(-.25)+1/6(-.25)+1/6(-.25)+1/6(-.25)+1/6(-.25) = -.25$ we will expect to lose on average 25 cents each time so no this is not a good bet.

\pagebreak
\subsection*{3. Linearity of Expectation}
For this problem we will solve the integral using these properties of variance and probability density functions. 
\begin{itemize}
  \item[] $ \int_{-\inf}^{\inf} p(x) dx = 1$ (1)
  \item[] $ \mu = \int_{-\inf}^{\inf} xp(x) dx$ (2)
  \item[] ${\sigma}^2 = \int_{-\inf}^{\inf} x^2p(x) dx - \mu^2 $ (3) 
\end{itemize}
First we will distribute the p(x) across the polynomial to get \[\int_{-\inf}^{\inf}ax^2p(x)+bxp(x)+cp(x) dx = a\int_{-\inf}^{\inf}x^2p(x) + b\int_{-\inf}^{\inf}xp(x) + c\int_{-\inf}^{\inf}p(x) \]
The answer falls out with the properties written above. The first term is $\sigma^2a = (1)a = a$ (3). Since we are given in the problem that $\mu=0$  the second term is $\mu = (0)b = 0$ (2). And the third term is just the integral of a pdf (1) times c which is c. As our final answer we have \[\int_{-\inf}^{\inf} p(x)(ax^2+bx+c) dx = a+c\]

\subsection*{4. Cumulative Density Functions / Calculus}
We need to use the cumulative density function and integration to find the area under the curve for when $0\leq x \leq \frac{1}{2}$ and $\frac{1}{2} \leq x \leq 1$ and add the two together to get the total area under the curve. \[\int_{0}^{1/2}4x dx + \int_{1/2}^{1}(-4x+4)dx = 1\] 
To find $P(a<X<b)$ we just put a and b into the integrals respectively depending on if their value is less than or greater than 1/2.
 (I think this is all we need to show)

\section*{Linear Algebra}
\subsection*{1. Transpose and Associative Property}
For this proof we will use properties of transpose and some basic linear algebra manipulations to get two identical scalars mutliplied by eachother.
\begin{align*}
  x^TBx = x^Tbb^Tx && \text{(definition of B given in problem)} \\
  =(b^Tx)^T(b^Tx) && \text{(definition matrix transpose)} \\
  =[(x^Tb)(x^Tb)]^T && \text{(definition matrix transpose)} \\
\end{align*}
since $x,b \in R^{d\times1}$ then $(x^Tb) \text{ has dimensions } (1 \times d)\times(d \times 1)$ which by definition of matrix multiplication has a dimension of $(1 \times 1)$ in otherwords just a scalar. Lets call this scalar $c$ Then we have $[(c)(c)]^T = (c^2)^T = c^2$ (by definition of transpose on a scalar) $c^2 = x^TBx \geq 0$ as was to be shown. 

\subsection*{2. Solving System of Linear Equation with Matrix Inverse}
\subsubsection*{(a)}

\begin{equation*}
  \begin{bmatrix}
    2 & 1 & 1\\
    4 & 0 & 2\\
    2 & 2 & 0
  \end{bmatrix}
  \begin{bmatrix}
    x_{1}\\
    x_{2}\\
    x_{3}
  \end{bmatrix}
  =
  \begin{bmatrix}
    3\\
    10\\
    -2
  \end{bmatrix}
\end{equation*}

\subsubsection*{(b)}
Using software we find the inverse of A to be: \(\begin{bmatrix}
  -1 & .5 & .5\\
  1 & -.5 & 0\\
  2 & -.5 & -1  
\end{bmatrix}\)
We can now use this information to find x by solving the equation $A^{-1}b = x$:
\begin{equation*}
  \begin{bmatrix}
    -1 & .5 & .5\\
    1 & -.5 & 0\\
    2 & -.5 & -1  
  \end{bmatrix}
  \begin{bmatrix}
    3\\
    10\\
    -2
  \end{bmatrix}
  =
  \begin{bmatrix}
    1\\
    -2\\
    3
  \end{bmatrix}
  = x
\end{equation*}

\section*{Proving Things}
\subsection*{1. Finding Maxima of a Function}
For this proof we will consider the graph $f(x) = ln(x)-x+1$ and show that its maximum value for $x>0$ is $f(1)=0$. First we take the derivitive of our function $f(x)$ and find where it is equal to 0. \[f'(x)=\frac{1}{x}-1 = 0 \implies x=1 \implies f(1)=0 \text{ is a maxima}\] 
We know there is a maxima or minima at $x=1$ to find which it is lets look at the second derivitive. \[f''(x)=-\frac{1}{x^2}\]
Since $f''(x) < 0$ for all $x>0$ we know that $x=1$ must be a maximum point on the interval $(0,\infty)$. Then $ln(x)-x+1 \leq 0 \implies ln(x) \leq x-1$ for all $x>0$ as was to be shown.

\subsection*{2. Proving Abstract Concepts}
We need to show that the divergence between $p_i$ and $q_i$ is non-negative given that $p_i \geq 0, q_i \geq 0, \text{for all } i \in \{0,1,2 \dots k\}$ 
i.e $\sum_{i=1}^{k}p_i\ln{\frac{p_i}{q_i}} \geq 0$
\begin{align*}
\begin{split}
  &\sum_{i=1}^{k}p_i\ln{\frac{p_i}{q_i}} \\ 
  &= \sum_{i=1}^{k}p_i(-\ln{\frac{q_i}{p_i}}) \\
  &\geq \sum_{i=1}^{k}p_i(-{\frac{q_i}{p_i}+1}) \text{         (this comes from problem 3.1 innequality changes due to negative)} \\
  &=-\sum_{i=1}^{k}q_i + \sum_{i=1}^{k}p_i \\
  &=-1+1=0
\end{split}
\end{align*}
  

\section*{Debriefing}
1. I spent probably 8 hours on this assignment \\
2. moderate difficulty\\
3. I worked on all of it alone except for some help from a math tutor on a few probability questions \\
4. 80\% \\
5. probably spent 6 hours working on the problems and another two hours trying to remember latex commands.

\end{document}